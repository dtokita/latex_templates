\documentclass[titlepage]{article}

%%% Packages begin %%%
\usepackage[margin=1in]{geometry}   % Used to generate the margins
\usepackage{fancyhdr}               % Used to generate the header
\usepackage{listings}               % Used to generate code
\usepackage{color}                  % Used to generate color
\usepackage{courier}
%%% Packages end %%%

\begin{document}

%%% Title page begin %%%
\title{\bf EE 406: Homework \#2}
\author{Dylan Tokita}
\date{\today}
\maketitle
%%% Title page end %%%

%%% Page configuration begin %%%
\pagenumbering{roman}               % Roman numerals for page number
\setcounter{page}{2}                % Start page count at 2
\parindent=0in                      % No indent on new paragraph
\parskip=8pt                        % 8 point space between paragraphs
\pagestyle{fancy}                   % Fancy header
\lhead{Dylan Tokita}
\chead{EE 406: Homework \#2}
%%% Page configuration end %%%

%%% Code configuration begin %%%
\definecolor{mygreen}{rgb}{0,0.6,0}
\definecolor{mygray}{rgb}{0.5,0.5,0.5}
\definecolor{mymauve}{rgb}{0.58,0,0.82}

\lstset{ %
  backgroundcolor=\color{white},   % choose the background color; you must add \usepackage{color} or \usepackage{xcolor}; should come as last argument
  basicstyle=\footnotesize\ttfamily,        % the size of the fonts that are used for the code
  breakatwhitespace=false,         % sets if automatic breaks should only happen at whitespace
  breaklines=true,                 % sets automatic line breaking
  captionpos=b,                    % sets the caption-position to bottom
  commentstyle=\color{mygreen},    % comment style
  deletekeywords={...},            % if you want to delete keywords from the given language
  escapeinside={\%*}{*)},          % if you want to add LaTeX within your code
  extendedchars=true,              % lets you use non-ASCII characters; for 8-bits encodings only, does not work with UTF-8
  %frame=single,	                   % adds a frame around the code
  keepspaces=true,                 % keeps spaces in text, useful for keeping indentation of code (possibly needs columns=flexible)
  keywordstyle=\color{blue},       % keyword style
  language=Octave,                 % the language of the code
  morekeywords={*,...},            % if you want to add more keywords to the set
  numbers=none,                    % where to put the line-numbers; possible values are (none, left, right)
  numbersep=5pt,                   % how far the line-numbers are from the code
  numberstyle=\tiny\color{mygray}, % the style that is used for the line-numbers
  rulecolor=\color{black},         % if not set, the frame-color may be changed on line-breaks within not-black text (e.g. comments (green here))
  showspaces=false,                % show spaces everywhere adding particular underscores; it overrides 'showstringspaces'
  showstringspaces=false,          % underline spaces within strings only
  showtabs=false,                  % show tabs within strings adding particular underscores
  stepnumber=2,                    % the step between two line-numbers. If it's 1, each line will be numbered
  stringstyle=\color{mymauve},     % string literal style
  tabsize=2,	                   % sets default tabsize to 2 spaces
  title=\lstname                   % show the filename of files included with \lstinputlisting; also try caption instead of title
}
%%% Code configuration end %%%

%%%%%%%%%%%%%%%%%%%%%%%%%%%%
%% --- Homework Start --- %%
%%%%%%%%%%%%%%%%%%%%%%%%%%%%

\section*{Problem 4.1}
How are the addresses decided for the following variables?

\begin{lstlisting}
#include <stdio.h>

int main () {
    printf("Hello World\n");
}
\end{lstlisting}

\end{document}
